\documentclass[12pt, a4paper]{report}
\usepackage[top=1cm, left=1cm, right=1cm]{geometry}

\usepackage[utf8]{inputenc}
\usepackage[russian]{babel}

\usepackage{array}
\newcolumntype{M}[1]{>{\centering\arraybackslash}m{#1}}

\usepackage{hyperref}
\hypersetup{
	colorlinks,
	citecolor=black,
	filecolor=black,
	linkcolor=black,
	urlcolor=black
}

\usepackage{sectsty}
\allsectionsfont{\centering}

\usepackage{indentfirst}
\setlength\parindent{24pt}

\usepackage{makecell}

\usepackage{amsmath}

\usepackage{tikz}
\usetikzlibrary{shapes.geometric, arrows}

\usepackage{listings}
\usepackage{xcolor}
\definecolor{codegreen}{rgb}{0,0.6,0}
\definecolor{codegray}{rgb}{0.5,0.5,0.5}
\definecolor{codepurple}{rgb}{0.58,0,0.82}
\definecolor{backcolour}{rgb}{0.95,0.95,0.92}
\lstdefinestyle{mystyle}{
    backgroundcolor=\color{backcolour},
    commentstyle=\color{codegreen},
    keywordstyle=\color{magenta},
    numberstyle=\normalsize\color{codegray},
    stringstyle=\color{codepurple},
    basicstyle=\ttfamily\footnotesize,
    breakatwhitespace=false,
    breaklines=true,
    captionpos=b,
    keepspaces=true,
    numbers=left,
    numbersep=5pt,
    showspaces=false,
    showstringspaces=false,
    showtabs=false,
    tabsize=2
}

\usepackage{graphicx}
\graphicspath{ {plots/pictures/} }

\usepackage{booktabs}
\usepackage{csvsimple}
\usepackage{longtable}
\usepackage{caption}

\begin{document}
	\begin{titlepage}
		\begin{center}
			\large \textbf{Министерство науки и высшего образования Российской Федерации} \\
			\large \textbf{Федеральное государственное бюджетное образовательное учреждение высшего образования} \\
			\large \textbf{«Российский химико-технологический университет имени Д.И. Менделеева»} \\

			\vspace*{6cm}
			\LARGE \textbf{ЛЕКЦИЯ №1}

			\vspace*{4cm}
			\begin{flushright}
				\Large
				\begin{tabular}{>{\raggedleft\arraybackslash}p{9cm} p{10cm}}
					Выполнил студент группы КС-36: & Золотухин Андрей Александрович \\
					Ссылка на репозиторий: & https://github.com/ \\
					& CorgiPuppy/ \\
					& chem-tech-control-sys-labs \\
				\end{tabular}
			\end{flushright}

			\vspace*{5cm}
			\Large \textbf{Москва \\ 2025}
		\end{center}
	\end{titlepage}

	\tableofcontents
	\thispagestyle{empty}
	\newpage

	\pagenumbering{arabic}

	\section*{Термодинамический анализ}
	\addcontentsline{toc}{section}{Термодинамический анализ}

	\subsection*{Уравнения сохранения}
	\addcontentsline{toc}{section}{Уравнения сохранения}
	\large
	Рассмотрю многофазную полидисперсную среду, где первая фаза (\underline{сплошная несущая}) - газ или жидкость, \textit{r} - фаза включений частиц, капель или пузырьков, размеры (\underline{объёмы}) которых \textit{r - dr}, \textit{r + dr}. \\
	Введу средние плотности в каждой точке объёма, занятого смесью:
	\begin{center}
		$\rho = \rho_{1} + \int_{0}^{R} \rho_{2}^{0} f(r)rdr$; $\>$ $\alpha = \alpha_{1} + \int_{0}^{R} rf(r)dr$; \\
		$\rho_{1} = \rho_{1}^{0} \alpha_{1}$; $\>$ $\alpha_{2} = \int_{0}^{R} rf(r)dr$,
	\end{center}
	где $\alpha_{i}$ - объёмные содержания фаз; $\rho$ - плотность смеси; $\rho_{i}^{0}$, $\rho_{i}$ - истинные и средние плотности фаз соответственно; $R$ - наибольший размер (\underline{объём}) включения; индекс $1$ относится к \underline{несущей} фазе, индекс $2$ - ко всей дисперсной (гетерогенной) фазе. \par
	Дисперсность гетерогенной фазы характеризуется функцией $f(r)$, так что $f(r)dr$ - число включений в единице объёма смеси, размеры (\underline{объёмы}) которых от $r$ до $r + dr$. \par
	Можно принять, что несущая фаза и все $r$-фазы - континуумы (материальная среда, которая считается сплошной, т.е. не разделенной на отдельные частицы или молекулы), заполняющие один и тот же объём и имеющие свою плотность, массу, скорость, температуру. \par
	Введение многоскоростного континуума необходимо, т.к. скорости относительного движения фаз в смеси по порядку могут быть равны скоростям их абсолютного движения. 1-ю фазу буду описывать моделью вязкой жидкости. В качестве тензоров поверхностных $\sigma_{1}^{kl}$, $\sigma_{2}^{kl}$ сил и тензоров вязких напряжений $\tau_{1}^{kl}$ приму:
	\begin{center}
		$\sigma_{1}^{kl} = -P_{1}\delta^{kl} + \tau_{1}^{kl}$, $\>$ $\sigma_{2}^{kl} = 0$, $\>$ $\tau_{1}^{kl} = \lambda \nabla \nu_{1} + 2\mu_{1}e_{1}^{kl}$,
	\end{center}
	где $\sigma^{kl}$ - символ Кронекера; $P_{1}$ - давление; $e_{1}^{kl}$ - тензор скоростей деформаций несущей фазы; $\lambda$, $\mu_{1}$ - коэффициенты вязкости. \par
	\textbf{Уравнение сохранения массы несущей фазы}:
	\begin{equation}\label{eq:conservationMassCarrierPhase}
		\frac{\partial \rho_{1}}{\partial t} + div(\rho_{1} \nu_{1}) = -\int_{0}^{R} \rho_{2}^{0} f \eta dr,
	\end{equation}
	где $\nu_{1}$ - средняя массовая скорость несущей фазы; $R$ - максимальный размер (\underline{объём}) включений; $\eta$ - наблюдаемая скорость изменения размера (\underline{объёма}) включения. \par
	В уравнении \eqref{eq:conservationMassCarrierPhase} член в правой части отражает \underline{суммарное влияние фазового перехода} \underline{на включениях}. \par
	\textbf{Уравнение баланса числа включений с учётом изменения объёма включения за счёт фазового перехода}:
	\begin{center}
		$\frac{\partial f}{\partial t} + div(f \nu_{2}) + \frac{\partial}{\partial r}(f \eta) = 0$,
	\end{center}
	где $\nu_{2}$ - средняя массовая скорость $r$-фазы; \par
	\textbf{Уравнения движения сплошной и $r$-фаз фаз}:
	\begin{center}
		$\rho_{1} \frac{d_{1} \nu_{1}}{dt} = -\alpha_{1} \nabla P_{1} + \nabla^{k} \tau_{1}^{k} - \int_{0}^{R} \rho_{2}^{0} f r f_{12} dr - int_{0}^{R} \rho_{2}^{0} f \eta (\nu_{2} - \nu_{1}) dr + F_{1}$, \\
		$\frac{D_{2}}{Dt}\nu_{2}(r) = - \nabla \frac{P_{1}}{\rho_{2}^{0}} + f_{12}(r) + F_{2}(r)$,	
	\end{center}
	где $P$ - давление $[\nabla P = (\frac{\partial P}{\partial x}, \frac{\partial P}{\partial y}, \frac{\partial P}{\partial z})]$; $F_{1}$, $F_{2}$ - массовые силы, действующие на несущую и $r$-фазу соответственно; $f_{12}$ - сила взаимодействия между несущей и $r$-фазами.
\end{document}
